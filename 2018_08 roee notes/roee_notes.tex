\documentclass{article}

\begin{document}

\section{Types of qubit}

Ions with a single electron in outer (valence) shell have a relatively simple level structure, so suitable. ``Singly ionised `alkali-like' ions" are the only ones currently used for QIP (Roee 2011).

Ground electronic state of valence electron: $^2 S_{1/2}$, where $S$ is the zero angular momentum orbital. The electron can be optically excited to the $P$ orbital via an electric dipole transition. 

Spin-orbit coupling $\rightarrow$ fine structure: \newline e.g., $P$ orbital splits to $P_{1/2}, P_{3/2}$. 

Odd isotopes or isotopes with non-zero nuclear spin $\rightarrow$ hyperfine structure: \newline e.g., $S_{1/2} \rightarrow S_{1/2}|F=0>, S_{1/2}|F=1>$.

(Small) magnetic field $\rightarrow$ Zeeman sub-levels: \newline e.g., $S_{1/2}|F=1> \rightarrow S_{1/2}|F=1, m_F = -1, 0, 1>$.

\textbf{Hierarchy of fine, hf, Zeeman structure? I.e., magnitudes for different atoms?}

Qubits are best encoded in a pair of levels that are very resilient/stable regarding decay and decoherence. Levels in the $P$ manifold have typical lifetimes of a few nS, so the usual choice is two $S_{1/2}$ states or an $S$ and $D$ superposition. Encoding a qubit in a pair of $S_{1/2}$ levels will have practically

\textbf{What makes a qubit resilient to decoherence? If lifetime is essentially infinite (e.g., S1/2 level), what different choices can be made vs decoherence...?}

\textbf{What does a qubit made up of an S and D superposition look like?}


\subsection{Zeeman qubit}

E.g., Yb 174 (I = 0). In a magnetic field the $S_{1/2}$ level is split into two states, corresponding to the electron spin pointing parallel and anti-parallel to the external magnetic field. The energy splitting between the two Zeeman levels is: \newline $g_s \mu_B B \approx 2 \cdot 10^{-23} \cdot B\approx 28$ MHz/mT.

Advantage of the Zeeman qubit discussed above: the Zeeman levels are the only two levels in the manifold, so ``optical pumping out of the qubit manifold does not occur". 

\textbf{Wait, what's optical pumping defined as? Why doesn't it affect us in this case?}

Disadvantages: energy difference betweens on magnetic field magnitude, which will be noisy and will result in dephasing. Also, if B field is small, the difference between the two Zeeman levels can be smaller than the $P$ level natural spectral width, so you can't do state selective fluorescence (if you shine a laser on either 0 or 1, both will be happily excited to $P$).

\textbf{What does dephasing specifically mean here...? Dephasing of what? Linked to coherence?}


\subsection{Hyperfine qubit}

Odd isotopes or when nuclear spin $I \neq 0$, the $S_{1/2}$ level is split into two hyperfine manifolds, $F = I \pm 1/2$. So for Yb 171, we end up with $S_{1/2}|F=0>$ and $S_{1/2}|F=1>$.  The hyperfine splitting (i.e., between F=0 and F=1) depends on the ground state hyperfine constant $A_{hf}$ and for the ions we care about, the splitting is in the few GHz range. 

If you apply a magnetic field, you remove the degeneracy of the Zeeman states in each hyperfine manifold, splitting them into $m_F$ states, where $m_F = -F, ..., F$. The Zeeman shift for small fields is $g_F \mu_B B m_F$ and can be calculated analytically using the Breit-Rabi formula.

Advantages of hyperfine qubit: you can choose a pair of levels such that their separation doesn't depend (to first order) on the magnetic field. E.g., the $S_{1/2}|F=0, m_F=0>$ and $S_{1/2}|F=1, m_F=0>$ levels in Yb 171. These are referred to as ``clock transition@ states.


\subsection{Optical qubit}

If your ion has a low-lying $D$ level, you could encode a qubit into levels connected by the $S \rightarrow D$ optical transition. The $D$ lifetimes are typically on the order of a second, so the fundamental limit to the optical qubit coherence time is long relative to typical operation time.

However, in contrast to Zeeman and hyperfine qubits (where the ``local oscillator"---the qubit energy splitting?---is in the rf or microwave range), optical qubits' local oscillator is in the optical range, i.e., a laser. Therefore the qubit coherence time is limited by the laser's finite linewidth***. \textbf{(Why?)} To get a good enough laser linewidth, e.g., comparable to the $D$ natural spectral width (about 1 Hz) requires laser frequency stability on the order of $10^{-14}$, which is non-trivial (ha. What's one of our laser's stability, out of curiosity?). This makes optical qubits less common.

\textbf{*** Why? How to think about this mechanically...? How does this affect coherence?}

For ions with $I \neq 0$, you'll get hyperfine splitting in both $S$ and $D$. The $P_{1/2}$ can't decay to $D_{5/2}$, so if you encode an optical qubit on the $S_{1/2} \rightarrow D_{5/2}$ transition, you can use state-selective fluorescence (an ion in $D$ won't scatter light).


\subsection*{Initialisation and detection}

Initialised using optical pumping, and detected usually using state-selective fluorescence: photons are state-selectively scattered if they are resonant with a transition connecting only one of the qubit states to a short lived state.

For Zeeman qubits (where the Zeeman splitting/B field is usually small compared to the P natural width) you first need to shelve one of the qubit states onto a ``spectrally distant meta-stable state" (e.g., one of the $D_{5/2}$ states), and then detect fluorescence.


\section{Single-qubit gates}

(Write out this section at home on pen and paper to get the proper hang of it?)

A single-ion qubit gate rotates the state vector(?) on the Bloch sphere.

The qubit levels are coupled by electromagentic radiation (e.g., laser light).

Three different cases:

\begin{itemize}
\item \textbf{Carrier transition:} the two qubit levels are coupled without changing the harmonic oscillator motional state.
\item \textbf{Red sideband transition:} when the ion qubit spin state is raised... a single h.o. quantum of motion is annihilated. (The interaction Hamiltonian for this is identical to the Jaynes-Cummings Hamiltonian.)
\item \textbf{Blue sideband transition:} when the ion qubit spin state is raised, a single h.o. quantum of motion is created. 
\end{itemize}

The RSB and BSD transitions both entangle the qubit's spin with the qubit's motional state.

\textbf{Lamb-Dicke regime:} when the deviation of the ion from its average position is much smaller than the radiation wavelength, or $\eta \sqrt{< ( \hat{a}^{\dagger} + \hat{a} )^2 >}$, ...the Rabi frequencies (the rate at which the qubit wavefunction cycles between the two qubit states) can be approximated simply, and they depend on $\eta$ and $n$, where $\eta = k x_0$ is the Lamb-Dicke parameter, where $k$ is the projection of \textbf{k} along the trap direction, and $x_0$ is the h.o. ground state width, and $n = 0,1, 2, ...$ describes the harmonic oscillator levels. 

\textbf{Which is biggest and how much typically do they vary by? 1\%, 10\%?}

The dependence of the Rabi frequencies on $\eta$ and $n$ can be understood by considering the effect of photon recoil on the overlap between different harmonic oscillator wavefunctions. (See examples...)

If you want to avoid entanglement of the qubit spin to the motion, you need to use on-resonance carrier transitions (i.e., not RSB or BSB transitions).

Now Roee's going to tell me about three different single-qubit gate coupling implementations, for the different ion qubit types.


\subsection{Magnetic dipole coupling}


\subsection{Two-photon Raman coupling}


\subsection{Optical quadrupole coupling}




\section{Two-qubit gates}



\end{document}